\usepackage{fancyhdr}

\renewcommand{\headrulewidth}{0.0pt}
\renewcommand{\footrulewidth}{0.0pt}
\setlength{\headheight}{25pt}

% Outside top: nothing
\fancyhead[LE,RO]{}
%%\fancyhead[LE,RO]{DRAFT \today}  % This puts a DRAFT note.

% Inside top: Chapter name, with link to table of contents.
\fancyhead[LO,RE]{\hyperlink{contents}{\slshape \leftmark}}

% Make header gray.
\definecolor{headergray}{rgb}{0.5,0.5,0.5}

% Make chapter header line: N. <name>
\renewcommand{\chaptermark}[1]{%
  \markboth{%
    \color{headergray}{%
      \thechapter.\ #1%
    }%
  }{}%
}

% NOTE: prevent the Bibliography from using a fake chapter number by redefining
% the chaptermark as such, right before the bibliography:
% \renewcommand{\chaptermark}[1]{%
%   \markboth{\color{headergray}{#1}}{}
% }

% Footer: gray page number.
\fancyfoot[C]{}

% Header and footer rule lines are colored the same, if used.
%% \renewcommand{\headrule}{
%%   \hbox to\headwidth {\color{headergray}\leaders\hrule height \headrulewidth\hfill}
%% }
%% \renewcommand{\footrule}{
%%   \hbox to\headwidth{\color{headergray}\leaders\hrule height \footrulewidth\hfill}
%% }

% Call: \pagestyle{fancy}

\fancypagestyle{plain}{%
  \fancyhf{}
  %\fancyfoot[C]{}
  \fancyfoot[RO,LE]{\thepage}
}

\fancypagestyle{thesis}{%
  \fancyhead{}%
  \fancyhead[RO,LE]{\hyperlink{contents}{\slshape \leftmark}}
  \fancyfoot[RO,LE]{\thepage}%
}
